\documentclass{article}
\usepackage{amsmath}
\usepackage{amssymb}
\usepackage{amsthm}
\usepackage{commath}
\newtheorem{theorem}{Theorem}
\DeclareMathOperator{\xc}{xc}
\DeclareMathOperator{\proj}{Proj}
\DeclareMathOperator{\Sl}{sl}
\begin{document}
\begin{theorem}
	Let $G=(V,E)$ be a graph, possibly with loops and parallel edges, with no $K_5$ minor with $\abs{V}=n$, $\abs{E}=m$. Then 

$$\xc(P^{\text{cut}}(G)) \in O(n^3)$$
\end{theorem}
\begin{proof}
For any graph $H=(V', E')$, define the polytope $R(H)$:

$$R(H) = \cbr{x\in \mathbf{R}^{E'}: \begin{array}{lll} x(F) - x(C-F) & \leq \abs{F} - 1 & \text{$C$ a cycle, $F\subseteq E'$, $\abs{F}$ odd} \\ 0 \leq x \leq 1 \end{array}}$$

In general, if $E' \subseteq E''$ where $E''$ is a multiset on $E$, then
$R(H) = \proj_{E'}((V', E''))$.

	In general, $P^{\text{cut}}(G) \subsetneqq R(G)$.
	Since $G$ has no $K_5$ minor, $P^{\text{cut}}(G) = R(G)$.
	Hence $P^{\text{cut}}(G) = \proj_{E} R((V, E\cup {V\choose 2}))=\proj_{E} R(K_n \cup E)$.
	(Note that it is possible that $E\not\subseteq {V\choose 2}$ since
	$G$ can have parallel edges and loops.)

	We exhibit a randomized communication protocol exchanging $3\log n + O(1)$ bits
	to compute the slack $\Sl((C,F), \delta(S))$ of
the inequality

$$x(F) - x(C-F)\leq \abs{F}-1$$

	for a cut $\delta(S)\subseteq E$, a (simple) cycle $C\subseteq E$, and an odd-size
subset $F\subseteq C$.

	If $C$ is a loop, then $\delta(S)\cap C = 0$. No communication is required.

	Now, assume $C$ is not a loop. Note that a simple cycle can only use
	one of any set of parallel edges.

Note that if $C$ has a chord $uv\in C$, then the above inequality is redundant.
Let $P_0, P_1$ be distinct $u,v$-paths in  $C$. Without loss
of generality, assume $\abs{P_0\cap F}$ is even. Then the above inequality
is the sum of the inequalities for $((P_0 + uv, (P_0 \cap F)\cup\cbr{uv}), \delta(S))$
and for $((P_1 + uv, P_1\cap F), \delta(S))$.

Hence the slack for a cycle with chords is the sum of the slacks of the cycles
between the chords. If there is an edge between every vertex, then 
	every cycle has enough chords
that the slack for a cycle can be computed by summing the slacks for
cycles of length three. As a lift for $\delta(S)$ 
from $\mathbf{R}^E$ to $\mathbf{R}^{E\cup{V\choose 2}}$,
we can use $\delta_{K_n\cup E}(S)$ since $P^{\text{cut}}(K_n \cup E) \subseteq R(K_n\cup E)$.

Now we describe the protocol. Beforehand, Alice and Bob agree
on a triangulation $C_1, C_2, \cdots, C_{\abs{C}-2}$ of every cycle $C$ in $G$,
and they also agree, for every odd-size $F\subseteq C$, on how to subdivide
$F$ and the chords $C_i\cap C_j$, $i\neq j$ into odd-size sets 
	$F_1\subseteq C_1; F_2\subseteq C_2;\cdots; F_{\abs{C}-2}\subseteq C_{\abs{C}-2}$ so that $\sum_{i=1}^{\abs{C}-2} \Sl((C_i, F_i),\delta(S)) = \Sl((C, F), \delta(S))$.

Alice receives a cycle $C$ and odd-size $F\subseteq C$. Bob receives
a cut $\delta(S)$. Alice chooses $r\in \sbr{\abs{C}-2}$ uniformly
at random, and sends the vertices $q_1,q_2,q_3$ of $C_r$ with $3\log n$ bits,
in that order.
Bob sends back three bits $b_{12}b_{23}b_{13}$ where $b_{st}$ is 1
if and only if $q_sq_t \in\delta(S)$.
Now, Alice knows $\abs{\delta{S}\cap F_r}$ and $\abs{\delta{S}\cap (C_r-F_r)}$,
so she can compute $\Sl((C_r, F_r),\delta(S))$. Alice outputs
$(\abs{C}-2)\Sl((C_r,F_r),\delta(S))$. The total cost
of the communication is $3\log n + O(1)$.

Alice has a uniform $\frac{1}{\abs{C}-2}$ probability of choosing
any particular triangle $C_r$, so in expectation, Alice computes
$\frac{1}{\abs{C}-2}\sum_{i=1}^{\abs{C}-2}\del{\abs{C}-2}\Sl((C_i,F_i),\delta(S)) = \Sl((C,F), \delta(S))$. 

Hence, there exists a communication protocol computing 
$\Sl((C,F), \delta(S))$ in $3\log n+O(1)$ bits. 
So $\xc(P^{\text{cut}}(G)) \in 2^{3\log n + O(1)} = O(n^3)$. 
\end{proof}

\begin{theorem}
	Let $G=(V,E)$ be planar. Then the perfect matching polytope has
	cubic extension complexity. That is,

	$$\xc(P^{\text{PM}}(G)) \in O(n^3)$$
\end{theorem}
\begin{proof}
	We exhibit a randomized communication protocol exchanging $3\log n+O(1)$
	bits to compute the slack $\Sl(S, M)$ of the inequality
	
	$$ x(\delta(S)) \geq 1$$

	for an odd-size $S\subseteq V$ and a perfect matching $M$.

	Since a perfect matching $M$ exists, $\abs{V}$ is even.
	Let $J$ be a $V$-join. Since $\abs{S}$ is odd, 
	$\abs{S}$ is a $V$-cut. 
	So, $\abs{\delta(S)\cap J}$ is odd.

	Let $Y = J\bigtriangleup M$. Since $J$ and $M$ are both $V$-joins,
	$Y$ is a $V\bigtriangleup V$-join or $\varnothing$-join.
	Equivalently, $Y$ induces an Eulerian subgraph of $G$.
	By planar duality, $Y$ induces a cut of $G^*$.

	Note that:
	\begin{align*}
		\Sl_\text{cut, $G$}(S, M) &= \abs{\delta(S)\cap M} - 1 \\
		&= \abs{\delta(S)\cap (Y \bigtriangleup J)} - 1 \\
		&= \abs{\delta(S)\cap Y} + \abs{\delta(S)\cap J} - 2\abs{\delta(S)\cap Y \cap J} - 1 \\
		&= \abs{\delta(S)\cap J} - 1 - \del{\abs{Y\cap(\delta(S)-(J\cap\delta(S)))} - \abs{Y\cap\delta(S)\cap J}} \\
		&= \Sl_\text{cut, $G^*$}((\delta(S), \delta(S)\cap J), Y) 
	\end{align*}

	where $\delta(S)$ is a cycle of $G^*$,
	and $\delta(S)\cap J$ is an odd-size subset.

	Hence, Alice and Bob need only compute the cut slack in $G^*$ as above
	in order to compute the perfect matching slack in $G$.
\end{proof}

	\begin{theorem}
	Let $G=(V,E)$ be a simple plane 2-edge-connected graph. Then,
		$$\xc(SEP(G)) \in O(n)$$
	\end{theorem}
\begin{proof}
	We exhibit a randomized communication protocol exchanging
	$\log n + O(1)$ bits to compute the slack $\Sl(S, X)$ of the
	inequality

	$$x(\gamma(S)) \leq \abs{S}-1$$

	for a set $S\subsetneqq V, 2\leq \abs{S} \leq \frac{\abs{V}}{2}$
	and $\delta(S)$ a minimal cut,
	and a vertex $X$ of $SEP(G)$.
	In advance, Alice and Bob agree on a vertex $v_\infty$.
	Alice and Bob also number the faces of $G$.
	
	Alice receives $S\subsetneqq V$. 
	If $v_\infty\in S$, Alice replaces $S$ with $V-S$.
	Bob receives $X\in SEP(G)$.

	Bob uses Pashkovich's compact spanning tree formulation to build 
	a fractional spanning arborescence of $(G-v_\infty)^*$
	in polynomial time, fixing the $x_e$ variables according to $X$.

	Let $F'$ be the set of faces $F(G[S])\cap F(G)$ 
	of  $G[S]$ that are also faces of $G$.
	By elementary manipulations, $x(\gamma(S))$ is equal to $z(\delta^{\text{out}}(F'))$, where a directed edge points "out" if its head is in a face of $G[S]$
	containing $v_\infty$ in its interior.
	Let $k=\abs{\delta^{\text{out}}(F')}$.

	Alice picks uniformly at random some directed edge $e\in \delta^{\text{out}}(F')$. 
	Let $u_0,u_1$ be the faces on the ends of $e$, where
	$u_0$ was lower numbered than $u_1$. Since
	$G$ is 2-edge-connected, $u_0\neq u_1$.
	Alice sends
	the name of $e$ and a bit $b$, where $e$ is pointing at $u_b$.

	Bob sends back a random bit $c$ where $c=1$ with probability $z_{e,u_b}$.
	Alice outputs $kc$.

	In expectation, this protocol computes $k\sum_{(e,u)\in\delta^{\text{out}}(F')} \frac{1}{k} z_{e,u} = z(\delta^{\text{out}}(F'))$. This protocol costs $\log n + O(1)$.
\end{proof}
\end{document}
